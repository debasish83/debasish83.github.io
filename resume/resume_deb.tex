% LaTeX resume using res.cls
\documentclass[margin]{res}
\usepackage[colorlinks]{hyperref}
%\usepackage{helvetica} % uses helvetica postscript font (download helvetica.sty)
%\usepackage{newcent}   % uses new century schoolbook postscript font 
\setlength{\textwidth}{5.1in} % set width of text portion

\begin{document}

% Center the name over the entire width of resume:
 \moveleft.5\hoffset\centerline{\large\bf Debasish Das}
% Draw a horizontal line the whole width of resume:
 \moveleft\hoffset\vbox{\hrule width\resumewidth height 1pt}\smallskip
% address begins here
% Again, the address lines must be centered over entire width of resume:
 \moveleft.5\hoffset\centerline{5069 Xavier Common}
 \moveleft.5\hoffset\centerline{Fremont, CA 94555}
 \moveleft.5\hoffset\centerline{Email: debasish.das83@gmail.com}
 \moveleft.5\hoffset\centerline{Phone: (408) 313-2465}
 \moveleft.5\hoffset\centerline{\href{https://debasish83.github.io}{Portfolio} 
 \href{https://www.linkedin.com/in/debasish83/}{LinkedIn}}

\begin{resume}

\section{OBJECTIVE} Technical Lead role focused on design and
implementation of large scale data pipelines and machine learning algorithms.

\section{EDUCATION} {\sl Doctoral in Computer Engineering} GPA : 3.9/4.0\\
Department of Electrical Engineering and Computer Science,\\
Robert R McCormick School of Engineering and Applied Sciences,\\
Northwestern University, Evanston, USA

{\sl Bachelor of Technology (B.Tech, Hons, 2004)} GPA : 8.3/10.0\\
Department of Computer Science and Engineering\\
India Institute of Technology, Kharagpur, India

\section{TECHNICAL SKILLS}
\begin{itemize} \itemsep -2pt
\item Languages: Scala, Java, C/C++, Javascript, Python, Tcl, Visual C++, Verilog
\item Skills: Spark, Lucene, Solr, Cassandra, HBase, Akka, YARN, HDFS, Distributed
  Systems, Machine Learning, Convex Optimization, Linear Programming,
  Multithreading, Algorithms, Data Structures, Automatic Differentiation
\end{itemize}

\section{OPEN SOURCE}
\begin{itemize} \itemsep -2pt
\item \href{https://github.com/Verizon/trapezium/tree/master/dal} {Data Access Object} for
  Spark, Lucene and Solr integration to support hybrid row/column storage and
  search for machine learning flows

\item \href{https://github.com/Verizon/trapezium/tree/master/framework}
  {Framework} for building Batch, Streaming and API workflows using Spark and Akka compute

\item Contributed \href{https://issues.apache.org/jira/browse/SPARK-3066} {recommendAll} feature to Spark MLlib

\item \href {https://issues.apache.org/jira/browse/SPARK-2426} {Constrained
  Matrix Factorization} for recommendation and topic modeling\\
Spark Summit 2014 \href {http://debasish83.github.io/quadprog/SparkSummit072014.html} {Presentation}

\item \href{https://issues.apache.org/jira/browse/SPARK-4823} {Row based similarity computation} and Twitter DimSum comparisons\\
Spark Meetup 2015 \href{http://debasish83.github.io/spark-meetup-july2015/mf-slides.pdf} {Presentation}

\item \href{https://github.com/scalanlp/breeze/tree/master/math/src/main/scala/breeze/optimize}{Quadratic
and Nonlinear Programming} solvers using ADMM

\item \href{https://github.com/embotech/ecos-java-scala} {Conic Solver} for JVM and Spark integration
\end{itemize}

\section{EXPERIENCE}

{\sl Distinguished Engineer / Software Architect} \hfill May 2013 - Present\\
Verizon, Palo Alto, CA, USA\\
Advisor: Ashok Srivastava, Professor Stephen Boyd
\begin{itemize} \itemsep -2pt

\item Advertising: \href{https://verizoninsights.verizon.com} {Verizon
  Audience Insight} for building Audience using Temporal, Location,
  Clickstream and CRM attributes from Verizon Wireless datasets and 
  serve ads on Oath properties (yearly revenue 25 Million)

\item Marketing: Near-RealTime Lookalike Modeling, Discriminant Analysis and Demand Forecasting for Audience\\
Spark Summit East 2017 
\href{https://spark-summit.org/east-2017/events/realtime-analytical-query-processing-and-predictive-model-building-on-high-dimensional-document-datasets-with-timestamps}{Presentation}\\
Spark Summit Europe 2016 \href{http://www.slideshare.net/SparkSummit/spark-summit-eu-talk-by-debasish-das-and-pramod-narasimha-68928564}{Presentation}

\item Platform-as-a-Service: Verizon Insight deployment for international carriers powered by \href{https://github.com/Verizon/trapezium}{trapezium} (yearly revenue 10 Million)

\item IoT Security: Streaming DDoS Detection on Verizon Wireline datasets using statistical, autoregressive and k-NN models\\
Hadoop Summit Tokyo 2016 \href{http://www.slideshare.net/HadoopSummit/near-realtime-network-anomaly-detection-and-traffic-analysis-using-spark-based-lambda-architecture}{Presentation}

\item Architecture validation for OLAP store using Solr, Druid and internally developed LuceneDAO 

\item Leading collaboration across Platform, Data Engineering and Data Science teams to hit timely release schedules

\item Provided technical guidance and coaching to developers, and conducted design discussion, code review and data validation.

\end{itemize}

{\sl Staff Research and Design Engineer} \hfill April 2012 - April 2013\\
Synopsys Inc, Mountain View, CA, USA
\begin{itemize} \itemsep -2pt
\item Generated machine learning models from post-routed sign-off timing data
\item Network flow solver benchmarking for System-On-Chip density optimization
\end{itemize}

{\sl Member of Consulting Staff} \hfill May 2011 - March 2012\\
Magma Design Automation, San Jose, CA, USA (Acquired by Synopsys Inc)
\begin{itemize} \itemsep -2pt
\item Developed algorithms to solve quadratic programs with nonlinear 
objectives derived from performance and legality constraints of VLSI circuit 
placement
\end{itemize}

{\sl Place and Route Development Engineer} \hfill September 2009 - May 2011\\
Mentor Graphics, San Jose, CA, USA
\begin{itemize} \itemsep -2pt
\item Developed multi-core graphical automatic differentiation engine for 
handling soft-max objectives in circuit placement algorithms
\item Developed Augmented Lagrangian based algorithms using L-BFGS to solve
  large scale nonlinear constrained optimization problems (tested upto 100M nodes and edges). 
\end{itemize}

{\sl Research Assistant} \hfill Spring 2005 - August 2009\\
EECS Department, Northwestern University, Evanston, IL, USA\\
Advisor: Professor Hai Zhou
\begin{itemize} \itemsep -2pt
\item Developed graph algorithms for static timing analysis with crosstalk
\item Developed lagrangian relaxation based sub-gradient optimizer for gate sizing problem
\end{itemize}

\section{ACHIEVEMENTS}
\begin{itemize} \itemsep -2pt
\item Verizon Spotlight Award for design and implementation of lambda architecture for streaming anomaly detection flows using Spark, Spark Streaming, Hive and Cassandra.
\item Place and Route Division, Mentor Graphics Excellence Award for architecting the multithreaded direct timing engine to drive timing driven placement.
\item Place and Route Group Excellence Award for developing the timing driven placement engine TESLA and replace the old offering with the new engine.  
\item Nominated for Intel Foundation PhD Fellowship from Robert R McCormick School of Engineering and Applied Sciences, Northwestern University for year 2007.
\item Awarded Walter P Murphy Fellowship by the Department of Electrical Engineering and Computer Science, Robert R McCormick School of Engineering and Applied 
  Sciences, Northwestern University for 2004-2005.
\end{itemize}

\section{IMMIGRATION STATUS}
Green Card

\section{SELECTED PUBLICATIONS}
\begin{itemize} \itemsep -2pt
\item Jia Wang, Debasish Das and Hai Zhou. Gate Sizing by Lagrangian Relaxation Revisited, IEEE/ACM 
  International Conference on Computer-Aided Design (ICCAD), San Jose, CA, 2007
\item Debasish Das, Kip Killpack, Chandramouli Kashyap, Abhijit Jas and Hai Zhou. Pessimism Reduction in
  Coupling Aware Static Timing Analysis Using Timing and Logic Filtering, IEEE/ACM Asia and South Pacific 
  Design Automation Conference (ASPDAC), Seoul, Korea, 2008. (Best Paper Award Nominee : 10/350)
\item Debasish Das, Jia Wang and Hai Zhou. iRetILP: An efficient incremental algorithm for min-period
  retiming under general delay model, Accepted to IEEE/ACM Asia and South Pacific 
  Design Automation Conference (ASPDAC), Hsinchu, Taiwan, 2010.
\item Ahmed Shebaita, Debasish Das, Dusan Petranovic and Yehea Ismail, A Novel Moment Based Framework For
  Accurate and Efficient Static Timing Analysis. Accepted to IEEE Transactions on Computer-Aided Design (TCAD).
\end{itemize}

\section{RELEVANT COURSES}
Design and Analysis of Algorithms, Advanced Algorithms, Seminar on Computer Security and Information Assurance, 
Random Processes in Communications and Control, Advanced Computer
Architecture, Numerical Methods for Engineers, 
Mathematical Programming, Programming and Data Structure, Discrete Structures,
Formal Language and Automata Theory, Artificial Intelligence, Applied Graph Theory, Probability and Statistics, Linear Algebra.

\section{ADDITIONAL ACTIVITIES}
\begin{itemize} \itemsep -2pt
\item Advisory Council member of \href{http://www.spark.tc/}{IBM Spark Technology Center} 
\item Reviewer for SIAM Data Mining, ICDM, TCAD, TVLSI, DAC, ICCAD, ISPD, ISQED, TAU and VLSI India
\end{itemize}

\end{resume}
\end{document}
