% LaTeX resume using res.cls
\documentclass[margin]{res}
\usepackage{hyperref}
%\usepackage{helvetica} % uses helvetica postscript font (download helvetica.sty)
%\usepackage{newcent}   % uses new century schoolbook postscript font 
\setlength{\textwidth}{5.1in} % set width of text portion

\begin{document}

% Center the name over the entire width of resume:
 \moveleft\hoffset\centerline{\large\bf Debasish Das}
% Draw a horizontal line the whole width of resume:
 \moveleft\hoffset\vbox{\hrule width\resumewidth height 1pt}\smallskip
% address begins here
% Again, the address lines must be centered over entire width of resume:
 \moveleft\hoffset\centerline{3560 Flora Vista Ave, Apt \# 223}
 \moveleft\hoffset\centerline{Santa Clara, CA 95051}
 \moveleft\hoffset\centerline{Email: debasish.das83@gmail.com}
 \moveleft\hoffset\centerline{(312) 354-0014}

\begin{resume}

\section{OBJECTIVE} A challenging software engineering position in the field
of large scale global optimization and mathematical modeling.

\section{EDUCATION} {\sl Doctoral in Computer Engineering} GPA : 3.9/4.0\\
Department of Electrical Engineering and Computer Science,\\
Robert R McCormick School of Engineering and Applied Sciences,\\ 
Northwestern University, Evanston, USA

{\sl Bachelor of Technology (B.Tech, Hons, 2004)} GPA : 8.3/10.0\\
Department of Computer Science and Engineering\\
India Institute of Technology, Kharagpur, India

\section{TECHNICAL SKILLS}
\begin{itemize} \itemsep -2pt
\item Languages: C++, Python, Java, Javascript, Tcl, Visual C++, Verilog
\item Tools: gdb, lex, yacc, gprof, CPLEX, AMPL, TLA+, Mentor Graphics EDA Suite, Synopsys DC/ICC
\end{itemize}

\section{OPEN SOURCE}
\begin{itemize} \itemsep -2pt
\item \href{www.stalkerbot.com}{StalkerBot} aggregates sentiments of Hacker News 
  user on a particular article based on the associated comments. 
  The application uses Diffbot API for tag generation and EffectCheck API for
  sentiment generation. Conceived and prototyped at AOL/StartX/Diffbot
  Hackathon.\href{https://github.com/debasish83/StalkerBot}{(Code)}
\item \href{https://github.com/debasish83/liblbfgs}{LBFGS solver for convex optimization}
\item \href{https://github.com/debasish83/sprinklersystem}{Java Simulator for
  Garden Sprinkler System}
\end{itemize}

\section{ACHIEVEMENTS}
\begin{itemize} \itemsep -2pt
\item Place and Route Division, Mentor Graphics Excellence Award for architecting the multithreaded direct timing engine to drive timing driven placement.
\item Place and Route Group Excellence Award for developing the timing driven placement engine TESLA and replace the old offering with the new engine.  
\item Nominated for Intel Foundation PhD Fellowship from Robert R McCormick School of Engineering and Applied Sciences, Northwestern University for year 2007.
\item Awarded Walter P Murphy Fellowship by the Department of Electrical Engineering and Computer Science, Robert R McCormick School of Engineering and Applied 
  Sciences, Northwestern University for 2004-2005.
\end{itemize}

\section{IMMIGRATION STATUS}
EB1-OR EAD card, AC21 Jan 2nd 2013, Green Card Expected Jan 2013

\section{EXPERIENCE}

{\sl Staff Research and Design Engineer} \hfill April 2012 - Present\\
Synopsys Inc, Mountain View, CA, USA\\
Advisor: Will Naylor
\begin{itemize} \itemsep -2pt
\item Generating timing models from post-routed sign-off timing data using
  physical models and convex machine learning techniques 
  (nonlinear regression). For solution QoR and runtime, compared
  effectiveness of Batch/Stochastic Gradient Descent, Conjugate Direction, 
  Conjugate Gradient and L-BFGS-B solvers.
\item Benchmarked internal network simplex solvers with cost scaling
  push-relabel algorithm and IBM CPLEX network optimizer. Network flow 
  solvers are used in density optimization for top level fragmented floorplans.
\end{itemize}

{\sl Member of Consulting Staff} \hfill May 2011 - Present\\
Magma Design Automation, San Jose, CA, USA (Acquired by Synopsys Inc)\\
Manager: Saurabh Adya
\begin{itemize} \itemsep -2pt
\item Developed algorithms to solve quadratic programs with nonlinear
  objectives derived from performance and legality constraints of VLSI circuit
  placement in Talus Vortex and Talus FX platform.
\item Developed timing objective driven placement algorithms for quadratic 
  programming based placement framework.
\item Used multithreading and streaming instructions to generate efficient
  implementation of the proposed optimization algorithms.
\end{itemize}

{\sl Place and Route Development Engineer} \hfill September 2009 - May 2011\\
Mentor Graphics, San Jose, CA, USA\\
Manager: Arash Hassibi and Phiroze Parakh
\begin{itemize} \itemsep -2pt
\item Developed an automatic differentiation engine (Direct Timing) for handling timing constraints in an analytical
timing framework. This engine is used to generate derivatives of timing QoR metrices (TNS and WNS) with respect to placement coordinates.
\item Developed the vectorized and scalable multithreaded version(3X faster on 1 thread and 6X faster on 4 threads than production Timer) of Direct Timing engine. 
\item Developed analytical placement algorithms that use Direct Timing to generate descent direction and guide analytical placer to produce better timing QoR(TNS and WNS) 
without affecting wirelength and routability metrics.
\item Developed Augmented Lagrangian based algorithms using core Quasi Newton Solvers (L-BFGS/L-BFGS-B) to solve large scale nonlinear/linear constrained 
optimization problems (tested upto 100M nodes and edges) involving wirelength, spread and direct timing modules. 
\end{itemize}

{\sl Research Assistant} \hfill Spring 2005 - August 2009\\
EECS Department, Northwestern University, Evanston, IL, USA\\
Advisor: Professor Hai Zhou
\begin{itemize} \itemsep -2pt
\item Developed algorithms and charge sharing based models to improve efficiency and accuracy of
  iterative Static timing analysis with crosstalk effects for early stages of design cycle. 
  Industry collaborators were Strategic CAD Lab, Intel and Faraday Technology.
\item Developed a sub-gradient optimizer for general gate sizing problem. Assuming the convexity
  on gate delays, developed an efficient method of feasible direction based solver to improve the 
  efficiency of general purpose sub-gradient optimizer (which is the state-of-the art algorithm for 
  sizing used in industry). This project is supported in parts by a NSF grant and a grant from 
  Intel. 
\item Developed an efficient incremental algorithm for minimum period retiming
  in general delay models for NSF supported project. Our algorithm achieved average 
  performance and memory improvements of 100X and 40X over previously proposed algorithm.
\end{itemize}

{\sl Research Assistant} \hfill Fall 2004 - Winter 2005\\
EECS Department, Northwestern University, Evanston, IL, USA\\
Advisor: Professor Seda Memik
\begin{itemize} \itemsep -2pt
\item Developed an algorithm to save leakage power in FPGAs using multiplexer shutdown.
\item Developed a tool based on VPR for the analysis of routing patterns in FPGAs.
\end{itemize}

{\sl Graduate Summer Intern} \hfill Fall 2007 - Winter 2008\\
Magma Design Automation, San Jose, CA, USA\\ 
Advisor: Dr. William Scott
\begin{itemize} \itemsep -2pt
\item Developed a current based model to improve efficiency and accuracy of crosstalk 
  induced delta delay computation for coupling analysis in late design stages. Model is generated
  from delay and slew tables rather than ECSM/CCSM data.
\item Reviewed state-of-the-art coupling analysis algorithms and found their ineffectiveness with
  extreme interconnect scaling for present and future process nodes.
\item Developed an algorithm based on the convexity of aggressor alignment curve to compute
  spice accurate crosstalk delta delay improving the ineffectiveness of present analysis algorithms.
\end{itemize}
    
{\sl Graduate Summer Intern} \hfill Summer 2007\\
Circuit Simulation Group, Intel Corporation, Santa Clara, CA, USA\\
Manager: Dr. Eric Grimme
\begin{itemize} \itemsep -2pt
\item Improved the complexity of general purpose transistor sizing by removing the constraints using
  lagrange multipliers from cell based sizer.
\item Developed a hybrid algorithm to merge sub-gradient optimization techniques used in standard cell
  based sizer with nonlinear optimization techniques used for transistor sizing.
\end{itemize}

{\sl Graduate Summer Intern} \hfill Summer 2006\\
Strategic CAD Lab, Intel Corporation, Hillsboro, OR, USA\\
Manager: Dr. Noel Menezes\\
Advisor: Kip Killpack and Dr. Chandramouli Kashyap
\begin{itemize} \itemsep -2pt
\item Developed an iterative algorithm for pessimism reduction in static timing analysis in presence 
  of crosstalk using logic and timing filtering. 
\item Derived a sensitivity based metric to select important aggressors for logic constraints analysis.
\end{itemize}

{\sl Graduate Summer Intern} \hfill Summer 2005\\
Calypto Design System Inc, San Jose, CA, USA\\
Manager: Dr. Sumit Roy\\
Advisor: Rajat Subhra Mukerjee and Abhishek Ranjan
\begin{itemize} \itemsep -2pt
\item Enhancements to SLEC Optimization engine like propagating constants across different types, 
  optimization of scan-latch designs and strength reductions.
\item Developed an efficient timing macromodeling algorithm for word level static timing analysis.
  For timing analysis of system level designs, fast generation of accurate timing macromodel 
  for system level blocks are essential. 
\end{itemize}

\section{REFERRED PUBLICATIONS}
\begin{itemize} \itemsep -2pt
\item Somsubhra Mondal, Debasish Das and Seda Memik. Hierarchical LUT Structures for Leakage Power 
  Reduction, Poster Paper, Proc. International Symposium on FPGAs (FPGA), Monterey, CA, 2005.
\item Debasish Das, Ahmed Shebaita, Hai Zhou, Yehea Ismail and Kip Killpack. FA-STAC: A Framework 
  for Fast and Accurate Static Timing Analysis with Coupling, IEEE International Conference on 
  Computer Design (ICCD), San Jose, CA, 2006.
\item Debasish Das, Ahmed Shebaita, Yehea Ismail, Hai Zhou and Kip Killpack. Nostra-XTalk: A Predictive 
  Framework for Accurate Static Timing Analysis in UDSM VLSI Circuits, ACM Great Lakes Symposium on VLSI (GLSVLSI), 
  Stresa, Italy, 2007.
\item Jia Wang, Debasish Das and Hai Zhou. Gate Sizing by Lagrangian Relaxation Revisited, IEEE/ACM 
  International Conference on Computer-Aided Design (ICCAD), San Jose, CA, 2007
\item Debasish Das, Kip Killpack, Chandramouli Kashyap, Abhijit Jas and Hai Zhou. Pessimism Reduction in
  Coupling Aware Static Timing Analysis Using Timing and Logic Filtering, IEEE/ACM Asia and South Pacific 
  Design Automation Conference (ASPDAC), Seoul, Korea, 2008. (Best Paper Award Nominee : 10/350)
\item Debasish Das, William Scott, Shahin Nazarian and Hai Zhou. An efficient Current Based Logic Cell
  Model for Crosstalk Delay Analysis, Accepted to International Symposium on Quality Electronic Design (ISQED), 
  San Jose, CA, 2009.
\item Debasish Das, Jia Wang and Hai Zhou. iRetILP: An efficient incremental algorithm for min-period
  retiming under general delay model, Accepted to IEEE/ACM Asia and South Pacific 
  Design Automation Conference (ASPDAC), Hsinchu, Taiwan, 2010.
\item Jia Wang, Debasish Das, and Hai Zhou. Gate Sizing by Lagrangian Relaxation Revisited. Accepted to
  IEEE Transactions on Computer-Aided Design (TCAD).
\item Debasish Das, Ahmed Shebaita, Hai Zhou, Yehea Ismail and Kip Killpack. FA-STAC: An Algorithmic
  Framework for Fast and Accurate Coupling Aware Static Timing Analysis. Accepted to IEEE
  Transactions on VLSI (TVLSI).
\item Debasish Das, Kip Killpack, Chandramouli Kashyap, Abhijit Jas and Hai Zhou, Pessimism Reduction 
  in Coupling Aware Static Timing Analysis Using Timing and Logic Filtering. Accepted to IEEE 
  Transactions on Computer-Aided Design (TCAD).
\item Ahmed Shebaita, Debasish Das, Dusan Petranovic and Yehea Ismail, A Novel Moment Based Framework For
  Accurate and Efficient Static Timing Analysis. Accepted to IEEE Transactions on Computer-Aided Design (TCAD).
\end{itemize}

\section{NONREFERRED PUBLICATIONS}
\begin{itemize} \itemsep -2pt
\item Debasish Das, Jia Wang and Hai Zhou. iRetILP: An efficient incremental algorithm for min-  retiming under general delay model, Accepted to ACM International Workshop on Timing Issues (TAU), 
  Austin, TX, 2009.
\item Debasish Das, William Scott, Shahin Nazarian and Hai Zhou. An Efficient Current Based Logic Cell
Model for Crosstalk Delay Analysis,ECM workshop, IEEE/ACM International Conference on Computer Aided Design,
San Jose, CA, 2008.
\item Abhijit Jas, Kip Killpack, Chandramouli Kashyap, Debasish Das and Hai Zhou. Using Boolean Satisfiability
  to Eliminate False Aggressor Combinations in Timing Analysis, International Test Synthesis Workshop, 
  San Antonio, Texas, USA, 2007.
\item Debasish Das. Symbolic Solver for live variable analysis of high level design languages. Accepted to 
  IEEE Computer Society Annual Symposium on VLSI, Karlsruhe, Germany, 2006.
\item Debasish Das, Abhishek Ranjan, Sumit Roy and Venky Ramachandran. Efficient Timing Macromodeling for 
  Word Level Static Timing Analysis, Internal Report, Calypto Design Systems Inc.
\end{itemize}

\section{SCORES} 
\begin{itemize} \itemsep -2pt
\item GRE: Quantative:800/800 Verbal:550/800 Analytical:4.5/6.0
\item TOEFL: 270/300, TSE: 45/60
\end{itemize}

\section{RELEVANT COURSES} {\sl Graduate Courses} \\
Computer Architecture, VLSI System Design, Introduction to VLSI CAD, 
Design and Analysis of High Speed ICs, Formal Techniques in Design 
and Verification of Digital Systems, Design and Analysis of Algorithms, 
Advanced Algorithms, Seminar on Computer Security and Information Assurance, 
Random Processes in Communications and Control – I, Advanced Computer 
Architecture – II, Numerical Methods for Engineers, Mathematical Programming\\

{\sl Undergraduate Courses} \hfill \\
Programming and Data Structure, Discrete Structures, Switching Circuits and 
Logic Design, Design and Analysis of Algorithms, Computer Organization and Architecture, 
Formal Language and Automata Theory, Operating Systems, Computer Networks, Microprocessors 
and Microcontrollers, Software Engineering, Electronic Design Automation, Artificial 
Intelligence, Compiler Construction, VLSI System Design, Embedded Systems, Applied Graph 
Theory, Low power circuits and systems, File Organization and Database Systems, Basic Electronics,
Electrical Technology, Electromagnetic Engineering, Semiconductor Devices, Probability and 
Statistics, Signals and Networks, Linear Algebra.

\section{ADDITIONAL ACTIVITIES}
\begin{itemize} \itemsep -2pt
\item Reviewer: TCAD, TVLSI, DAC, ICCAD, ISPD, ISQED, TAU, VLSI India
\item Teaching Assistant for ECE203 (Introduction to Computer Engineering) Fall, Spring 2005, 
  ECE231 (Advanced Programming for Computer Engineers) Winter 2006, ECE357 (Introduction 
  to VLSI CAD) Fall 2006, EECS366 (Design and Analysis of Algorithms) Winter 2006.
\end{itemize}

\section{REFERENCES}
\begin{itemize} \itemsep -2pt
\item Dr. Premal Buch, Vice President, Altera Corporation, San Jose, USA\\
  Email : pbuch@altera.com
\item Dr. Phiroze Parakh, MTS, Google, Mountain View, USA\\
  Email : phiroze.parakh@gmail.com
\item Dr. Arash Hassibi, Software Engineer, iCelero Technologies\\
  Email : arash.hassibi@gmail.com
\item Dr. Sumit Roy, Group Director, Synopsys Inc, Mountain View, USA\\
  Email : sroy@synopsys.com
\item Professor Hai Zhou, Department of EECS, Northwestern University, USA\\
  Email : haizhou@eecs.northwestern.edu
\item Professor Yehea Ismail, Department of EECS, Northwestern University, USA\\
  Email : ismail@eecs.northwestern.edu
\item Dr. Noel Menezes, Manager, Strategic CAD Lab, Intel Corporation, USA\\
  Email : noel.menezes@intel.com
\item Dr. Eric Grimme, Manager, Circuit Simulation Group, Intel Corporation, USA\\
  Email : eric.grimme@intel.com
\item Professor Ajit Pal, Department of CSE, IIT Kharagpur, India\\
  Email : apal@cse.iitkgp.ac.in
\end{itemize}

\end{resume}
\end{document}
